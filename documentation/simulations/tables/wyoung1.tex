\documentclass{article}
\usepackage{booktabs}
\usepackage{tabularx}
\usepackage[margin=1in]{geometry}
\begin{document}

\begin{table}[tbp] \centering
\newcolumntype{C}{>{\centering\arraybackslash}X}

\caption{Family-wise rejection proportions at \(\alpha = 0.05\)}
\label{tab:wyoung1}
\begin{tabularx}{\linewidth}{lCCCC}

\toprule
&{(1)}&{(2)}&{(3)}&{(4)} \tabularnewline \midrule
{Adjustment method}&{Normal errors}&{Multiple subgroups}&{Correlated errors}&{Lognormal errors} \tabularnewline
\midrule \addlinespace[\belowrulesep]
Unadjusted&0.398&0.387&0.685&0.577 \tabularnewline
Bonferroni-Holm&0.040&0.047&0.344&0.234 \tabularnewline
Sidak-Holm&0.040&0.051&0.347&0.237 \tabularnewline
Westfall-Young&0.041&0.045&0.513&0.058 \tabularnewline
\midrule Num. observations&100&100&100&100 \tabularnewline
Num. hypotheses&10&10&10&10 \tabularnewline
Hypotheses are true&Y&Y&N&Y \tabularnewline
\bottomrule \addlinespace[\belowrulesep]

\end{tabularx}
\begin{flushleft}
\footnotesize Notes: Table reports the fraction of 2,000 simulations where at least one null hypothesis in a family of 10 hypotheses was rejected. All hypotheses are true for the simulations reported in columns (1), (2), and (4), i.e., lower rejection rates are better. All hypotheses are false for the simulation reported in column (3), i.e., higher rejection rates are better. The Westfall-Young correction is performed using 1,000 bootstraps.
\end{flushleft}
\end{table}
\end{document}
