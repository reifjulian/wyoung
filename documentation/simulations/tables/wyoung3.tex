\documentclass{article}
\usepackage{booktabs}
\usepackage{tabularx}
\usepackage[margin=1in]{geometry}
\begin{document}

\begin{table}[tbp] \centering
\newcolumntype{R}{>{\raggedleft\arraybackslash}X}
\newcolumntype{L}{>{\raggedright\arraybackslash}X}
\newcolumntype{C}{>{\centering\arraybackslash}X}

\caption{Family-wise rejection proportions at \(\alpha = 0.05\), when testing hypotheses with multiple regressors or restrictions}
\label{tab:wyoung3}
\begin{tabularx}{\linewidth}{@{}lCCC@{}}

\toprule
&{(1)}&{(2)}&{(3)} \tabularnewline \midrule
{Adjustment method}&{Multiple regressors}&{Linear restriction}&{Nonlinear restriction} \tabularnewline
\midrule \addlinespace[\belowrulesep]
Unadjusted&0.634&0.440&0.435 \tabularnewline
Bonferroni-Holm&0.043&0.052&0.064 \tabularnewline
Sidak-Holm&0.045&0.052&0.066 \tabularnewline
Westfall-Young&0.041&0.051&0.062 \tabularnewline
\midrule Num. observations&100&100&100 \tabularnewline
Num. hypotheses&20&10&10 \tabularnewline
\bottomrule \addlinespace[\belowrulesep]

\end{tabularx}
\\ \parbox{\linewidth}{\footnotesize Notes: Table reports the fraction of 2,000 simulations where at least one null hypothesis in the family was rejected. All null hypotheses are true, i.e., lower rejection rates are better. Section \ref{SS-multiple regressors} describes the data-generating process used in Column (1). Section \ref{SS-combination} describes the data-generating process used in Columns (2) and (3). The Westfall-Young correction is performed using 1,000 bootstraps.}
\end{table}
\end{document}
