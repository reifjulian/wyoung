\documentclass{article}
\usepackage{booktabs}
\usepackage{tabularx}
\usepackage[margin=1in]{geometry}
\begin{document}

\begin{table}[tbp] \centering
\newcolumntype{C}{>{\centering\arraybackslash}X}

\caption{Family-wise rejection proportions at \(\alpha = 0.05\), when testing hypotheses with multiple restrictions}
\label{tab:wyoung3}
\begin{tabularx}{\linewidth}{lCC}

\toprule
&{(1)}&{(2)} \tabularnewline \midrule
{Adjustment method}&{Linear restriction}&{Nonlinear restriction} \tabularnewline
\midrule \addlinespace[\belowrulesep]
Unadjusted&0.440&0.435 \tabularnewline
Bonferroni-Holm&0.052&0.064 \tabularnewline
Sidak-Holm&0.052&0.066 \tabularnewline
Westfall-Young&0.051&0.063 \tabularnewline
\midrule Num. observations&100&100 \tabularnewline
Num. hypotheses&10&10 \tabularnewline
\bottomrule \addlinespace[\belowrulesep]

\end{tabularx}
\begin{flushleft}
\footnotesize Notes: Table reports the fraction of 2,000 simulations where at least one null hypothesis in a family of 10 hypotheses was rejected. All null hypotheses are true, i.e., lower rejection rates are better. The Westfall-Young correction is performed using 1,000 bootstraps.
\end{flushleft}
\end{table}
\end{document}
