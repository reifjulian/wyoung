\documentclass{article}
\usepackage{booktabs}
\usepackage{tabularx}
\usepackage[margin=1in]{geometry}
\begin{document}

\begin{table}[tbp] \centering
\newcolumntype{R}{>{\raggedleft\arraybackslash}X}
\newcolumntype{L}{>{\raggedright\arraybackslash}X}
\newcolumntype{C}{>{\centering\arraybackslash}X}

\caption{Family-wise rejection proportions at \(\alpha = 0.05\), when treatment is randomized}
\label{tab:wyoung4}
\begin{tabularx}{\linewidth}{@{}lCCC@{}}

\toprule
&{(1)}&{(2)}&{(3)} \tabularnewline \midrule
& \multicolumn{3}{c}{Method of random assignment} \tabularnewline \cmidrule{2-4} 
{Adjustment method}&{Individual}&{Stratified}&{Clustered} \tabularnewline
\midrule \addlinespace[\belowrulesep]
Unadjusted&0.392&0.403&0.391 \tabularnewline
Bonferroni-Holm&0.051&0.041&0.045 \tabularnewline
Sidak-Holm&0.054&0.042&0.045 \tabularnewline
Westfall-Young (bootstrap)&0.053&0.064&0.043 \tabularnewline
Westfall-Young (permutation)&0.052&0.043&0.043 \tabularnewline
\midrule Num. observations&100&100&1,000 \tabularnewline
Num. hypotheses&10&10&10 \tabularnewline
\bottomrule \addlinespace[\belowrulesep]

\end{tabularx}
\\ \parbox{\linewidth}{\footnotesize Notes: Table reports the proportion of 2,000 simulations where at least one null hypothesis in the family was rejected. All null hypotheses are true, so lower rejection rates indicate better performance. In column (1), individuals are randomly assigned to treatment with a probability of 0.5. In column (2), assignment is stratified into 10 equally sized strata. In column (3), treatment is assigned at the cluster level, with 100 clusters of 10 observations each. The Westfall-Young adjustments are applied using 1,000 bootstraps/permutations.}
\end{table}
\end{document}
